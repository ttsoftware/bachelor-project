\documentclass[12pt]{article}

\usepackage[utf8]{inputenc}
\usepackage[english]{babel}
\usepackage{amssymb}
\usepackage{amsfonts}
\usepackage{amsmath}
\usepackage{graphicx}
\usepackage[margin=1in]{geometry}
\usepackage[hidelinks]{hyperref}
\usepackage{float}
\usepackage[danish]{varioref}
\usepackage{multirow}
\usepackage{hhline}
\usepackage{inconsolata}
\usepackage{etoolbox}
\usepackage[usenames,dvipsnames]{xcolor}
\usepackage{tikz}
\usetikzlibrary{positioning,shapes, shadows, arrows}
\usepackage{listings}
\usepackage{pgfgantt}

\setlength\parindent{0pt}
\usepackage[parfill]{parskip}

\definecolor{dark-blue}{HTML}{000080}
\definecolor{dark-green}{HTML}{008000}
\definecolor{pale-purple}{HTML}{94558D}
\definecolor{dark-purple}{HTML}{0000AA}
\definecolor{regular-purple}{HTML}{660099}
\definecolor{magenta}{HTML}{B200B2}
\definecolor{light-gray}{HTML}{FAFAFA}
\definecolor{dark-gray}{HTML}{2D2D2D}
\definecolor{comment}{HTML}{808080}
\definecolor{digit}{HTML}{0000FF}

\newcommand*{\FormatDigit}[1]{\textcolor{digit}{#1}}

\lstset{
	language=Python,
	prebreak=\raisebox{0ex}[0ex][0ex]{\ensuremath{\color{red}\space\hookleftarrow}},
	basicstyle=\footnotesize\ttfamily,
	%
	literate=%
    	{0}{{\FormatDigit{0}}}{1}%
        {1}{{\FormatDigit{1}}}{1}%
        {2}{{\FormatDigit{2}}}{1}%
        {3}{{\FormatDigit{3}}}{1}%
        {4}{{\FormatDigit{4}}}{1}%
        {5}{{\FormatDigit{5}}}{1}%
        {6}{{\FormatDigit{6}}}{1}%
        {7}{{\FormatDigit{7}}}{1}%
        {8}{{\FormatDigit{8}}}{1}%
        {9}{{\FormatDigit{9}}}{1}%
        {.0}{{\FormatDigit{.0}}}{2}% Following is to ensure that only periods
        {.1}{{\FormatDigit{.1}}}{2}% followed by a digit are changed.
        {.2}{{\FormatDigit{.2}}}{2}%
        {.3}{{\FormatDigit{.3}}}{2}%
        {.4}{{\FormatDigit{.4}}}{2}%
        {.5}{{\FormatDigit{.5}}}{2}%
        {.6}{{\FormatDigit{.6}}}{2}%
        {.7}{{\FormatDigit{.7}}}{2}%
        {.8}{{\FormatDigit{.8}}}{2}%
        {.9}{{\FormatDigit{.9}}}{2}%
        %{,}{{\FormatDigit{,}}{1}% depends if you want the "," in color
        {\ }{{ }}{1}% handle the space
		{æ}{{\ae}}1
        {ø}{{\o}}1
        {å}{{\aa}}1
        {Æ}{{\AE}}1	
        {Ø}{{\O}}1
        {Å}{{\AA}}1,
	%
	%emph={},
	otherkeywords={},
	keywords=[2]{self},
	keywords=[3]{__init__},
	keywords=[4]{object},
	keywords=[5]{encoding, flags},
	keywords=[6]{reduce,list,enumerate,len,map,range,sorted,None,super,@staticmethod},
	%
	keywordstyle=\bfseries\color{dark-blue},
	keywordstyle={[2]\color{pale-purple}},
	keywordstyle={[3]\color{magenta}},
	keywordstyle={[4]\color{dark-blue}},
	keywordstyle={[5]\color{regular-purple}},
	keywordstyle={[6]\color{dark-purple}},
    commentstyle=\itshape\color{comment},
    identifierstyle=\color{black},
	stringstyle=\bfseries\color{dark-green},
	%emphstyle=\color{dark-purple},
	%
	numbers=left, % where to put the line-numbers
	numberstyle=\ttfamily\color{dark-gray},
	numbersep=5pt, % how far the line-numbers are from the code
	stepnumber=1,
	showstringspaces=false,
	backgroundcolor=\color{light-gray},
	tabsize=4,
	captionpos=b, % sets the caption-position to bottom
	breaklines=true % sets automatic line breaking
}

\linespread{1.3}

\begin{document}

\begin{titlepage}
    \vspace*{\fill}
    \begin{center}
      {\Huge Synopsis for Bachelorproject}\\[0.7cm]
      {\Large Regular Expression Matching In Genomic Data}\\[0.4cm]
      {\large Rasmus Haarslev - nkh877}\\
      {\large Troels Thomsen - qvw203}\\[0.4cm]
      {Supervisiors: Rasmus Fonseca \& Fritz Henglein}\\
      {\small 23. Februar 2015}\\[0.3cm] 
      {\small Department of Computer Science}\\
      {\small University of Copenhagen}
    \end{center}
    \vspace*{\fill}
\end{titlepage}	

\clearpage

\thispagestyle{empty}

\newpage

\section{Problem definition}

We wish to determine the possibility of converting sequence analysis patterns used for scan-for-matches\cite{scan-for-matches}, into regular expressions\cite{crash-course-regex} and test their efficiency against the KMC\cite{kmc-website} engine.

Specifically we wish to solve the following problems:

\begin{itemize}
	\item Is it possible to programatically convert patterns used by the scan-for-matches program into regular expressions for the KMC engine? If not all patterns used by scan-for-matches then which ones?
	\item Is it possible to achieve speeds matching or exceeding scan-for-matches with the generated regular expressions and the KMC engine?
	\item Are there features missing from the KMC engine (such as backtracking), which if they were present would yield better performance in the case of these specific patterns?
\end{itemize}

\subsection{Limits}

\begin{itemize}
	\item We will not attempt to modify the KMC engine.
\end{itemize}

\newpage

\section{Motivation}

Institute for Bioinformatics have a allocated, and are still allocating, a lot of DNA sequencing data. Currently they have around a total of 2 petabytes of data. These sequences of DNA contain a lot of information, but searching through the data currently uses the scan-for-matches program, which while performing very well, is not very user friendly and have some unfortunate limitations when running many consecutive scans, since it performs I/O operations for every run.

Recently Fritz' group have developed a regular expression engine called KMC, which so far have performed five times better than current industry standard engines. Since scan-for-matches outperforms NR-grep\cite{nrgrep}, we are hoping that optimized regular expressions running on KMC will be able to outperform NR-grep and subsequently scan-for-matches.

If we could achieve a performance improvement over scan-for-matches, it would greatly benefit the bioinformatics team. As such we see this as a chance to make a unique contribution to ongoing and future research projects, while at the same time providing a chance for the KMC team to have their engine tested in a new scenario.

\newpage

\section{Tasks and Schedule}

\begin{itemize}
	\item Develop a standalone Ruby and C application as a solution to the problem.
	\begin{itemize}
		\item \textbf{Product:} A fully functional Ruby/C application, that can translate scan-for-matches patterns into regular expressions, understood by the KMC engine.
		\item \textbf{Resource demands:} Laptops, our contact persons with insight in the KMC engine, testing data in the fasta format, and the KMC engine.
		\item \textbf{Dependencies:} Problem formulation
		\item \textbf{Time demands:} 5 weeks
	\end{itemize}
	
	\item Test and analyse the efficiency of our application compared to scan-for-matches.
	\begin{itemize}
		\item \textbf{Product:} An extensive analysis of our application, with suggestions for possible improvements.
		\item \textbf{Resource demands:} Laptops, testing data in the fasta format, and the KMC engine.
		\item \textbf{Dependencies:} The application that we developed.
		\item \textbf{Time demands:} 5 weeks (In parallel with development)
	\end{itemize}
	
	\item Improve upon our application based on our tests and analysis.
	\begin{itemize}
		\item \textbf{Product:} An improved application for translating scan-for-matches patterns into regular expressions, understood by the KMC engine.
		\item \textbf{Resource demands:} Laptops, our application, testing data in the fasta format, and the KMC engine.
		\item \textbf{Dependencies:} The written analysis.
		\item \textbf{Time demands:} 3 weeks
	\end{itemize}
	
	\item Analyse the KMC engine, and find possible improvements to the engine.
	\begin{itemize}
		\item \textbf{Product:} Analysis of the KMC engine as well as suggestions for possible improvements.
		\item \textbf{Resource demands:} Laptops, and the KMC engine.
		\item \textbf{Dependencies:} None.
		\item \textbf{Time demands:} 4 weeks
	\end{itemize}
\end{itemize}

\begin{tabular}{|r|cccccccccccccccc|}
\hline
\textbf{week} & 8 & 9 & 10 & 11 & 12 & 13 & 14 & 15 & 16 & 17 & 18 & 19 & 20 & 21 & 22 & 23 \\
\hline
Development & X & X & X & X & X & & - &&&&&&&&& \\
Test \& analysis & & & X & X & X & X & - & X &&&&&&&& \\
Improvement &&&&&& X & - & X & X &&&&&&& \\
KMC analysis &&&&&&& - &&&& X & X & X & X && \\
Midway report &&&&&&& - & X & X & X &&&&&& \\
Final report &&&&&&& - &&&&& X & X & X & X & X \\
\hline
\end{tabular}

%\begin{ganttchart}[hgrid, vgrid, x unit=0.15cm]{55}{159}
	%\gantttitlecalendar{month=shortname, week=9} \\
	%\gantttitlelist{9,...,23}{7} \\
	%\ganttset{progress label text={}, link/.style={black, -to}}
	%\ganttgroup{Objective 1}{2}{15} \\
	%\ganttbar[progress=4, name=T1A]{Task A}{25}{46} \\
	%\ganttlinkedbar[progress=0]{Task B}{10}{24} \\
	
	%\ganttgroup{Objective 2}{1}{5} \\
	%\ganttbar[progress=15, name=T2A]{Task A}{5}{36} \\
	%\ganttlinkedbar[progress=0]{Task B}{20}{78} \\
	
	%\ganttgroup{Objective 3}{2}{58} \\
	%\ganttbar[progress=0]{Task A}{11}{22} \\
	%\ganttset{link/.style={OliveGreen}} \\
%\end{ganttchart}

	
\newpage

\bibliographystyle{plain}
\nocite{*}	
\bibliography{litterature}

\end{document}
